\documentclass[main.tex]{subfiles}
\begin{document}
\newpage
\section{Telegrafenrauschen in Samarium-Erbium-Ferrit}

Der Orthoferrit besteht aus vier Untergittern, welche jeweils eine andere Magnetisierung \(\vec{\mu}_i\) besitzen (Normiert \(\vec{S}_i\)). Im nachfolgenden Bereich wird die normierte resultierende Magnetisierung (Überlagerung der vier Untergitter) verwendet.

\begin{align}
    \mu_{1} &=  \mu_{2} = \mu_{3} = \mu_{4} = \num{3.66}\mu_B \\
    \vec{m} &= \frac{1}{4} \sum_{i=1}^4 \vec{S}_i
\end{align}

Das Telegrafenrauschen tritt am untersuchten Reorientierungsübergang in c-Richtung auf, weshalb nur diese Komponente betrachtet wird. 

\todo{plot Reorientierungsübergang?}

Eine Simulation, von einem \(128 \times 128 \times 128\) Gitter mit einem simulierten Zeitraum von \SI{1.2}{\nano\second} mit einer numerischen Schrittweite von \SI{20}{\femto\second} benötigt etwa drei Tage Rechenzeit auf einer Nvidea RTX 2080 Ti Grafikkarte. 

In diesem Simulationszeitraum sind im optimalen Temperaturbereich (um mehrere eindeutige Schaltvorgänge zu beobachten) etwa 15-25 Schaltvorgänge pro Simulation zu beobachten vgl. \cref{fig:bsp-run}

Um mehr Schaltvorgänge pro Simulation betrachten zu können, wurde der Simulationszeitraum teilweise verdoppelt.

\begin{figure}[h]
    \centering
    \includegraphics{bilder/plots/theo-vis/example-telegraph-sim.pdf}
    \caption{Einzelne Simulationskurve von \(m_c\) bei \(T \approx \SI{302}{\kelvin}\). Im Zeitraum \(<0\) ist hierbei der Einschwingvorgang zu sehen \todo{Bezeichnung für die zwei Niveaus/Zustände?}
    }\label{fig:bsp-run}
\end{figure}

\todo{erklärung warum verschiedene Runs einfach hintereinander hängen geht}


Die Mittleren Aufenthaltszeiten (MDT) in den Beiden Niveaus befinden sich in der gleichen größenordnung, wie der Simulationszeitraum. Um auch bei weniger optimalen Daten, wie in \cref{fig:extraktion-tgr}, eine gute Extraktion des Telegrafenrauschens zu ermöglichen, wurden teilweise die Signale mehrerer Simulationsläufe hintereinander gehängt, um ein langes Signal zu erhalten.

Da aber nicht Zufälligerweise alle Simulationsläufe immer im gleichen Niveau starten und aufhören (so lange die Wahrscheinlichkeit für einen Zustand nicht deutlich höher ist), werden hierbei teilweise zusätzliche Schaltvorgänge hinzugefügt.

\todo{warum ist das meißtens kein problem?}

Andere Möglichkeiten um die Rechenzeit zu verringern wären zum Beispiel eine geringere Gittergröße, oder eine größere Zeitschrittweite.

Die Gittergröße ist allerdings bereits recht klein und eine größere Zeitschrittweite würde die Genauigkeit der Simulation verringern, sodass die Ergebnisse nicht mehr mit den experimentellen Daten vergleichbar wären. Bei einer Schrittweite von \SI{100}{\femto\second} (wahrscheinlich aber schon früher) tritt das Telegrafenrauschen garnicht mehr auf und der Betrag der Magnetisierung ist deutlich größer.


\todo{andere möglichkeiten die Rechenzeit zu verringern}

\todo{was passiert im Einschwingvorgang?}

\subsection{Extraktion des Telegrafenrauschens}

\begin{figure}[H]
    \centering
    \includegraphics{bilder/plots/Bz_0mT/mc_fit_hist_part2_26.03meV.pdf}
    \caption{Extraktion des Telegrafenrauschen mithilfe des Algorithmus aus \cref{algo}. Das bereinigte Telegrafenrauschen ist im linken Plot in rot zu sehen. Die Differenz zwischen bereinigtem Telegrafenrauschen und dem rohen Signal ist in grün zu sehen.}\label{fig:extraktion-tgr}
\end{figure}

\todo{Gauß vs. Lorenzverteilung zum Fitten?}

Die Aufenthaltswahrscheinlichkeit zwischen den zwei Zuständen ist deutlich größer, als bei zwei überlagerte Gaußverteilungen anzunehmen wäre. Bei Zwei Lorenzverteilungen ist die Aufenthaltswahrscheinlichkeit im Randbereich jedoch gar nicht passend.

\begin{figure}[H]
    \centering
    \includegraphics{bilder/plots/theo-vis/hist_fit_comp.pdf}
    \caption{Vergleich von verschiedenen Verteilungsfunktionen um die Wahrscheinlichkeitsdichte zu fitten}\label{fig:fit_comp}
\end{figure}

\subsection{Dominanz des Telegrafenrauschens}

\begin{figure}[H]
    \centering
    \includegraphics{bilder/plots/Bz_0mT/spectral_power_densities_26.03meV.pdf}
    \caption{Spektrale Leistungsdichten von ursprünglichem und bereinigtem Telegrafenrauschen und Differenz. Simulation bei \SI{26.04}{\milli\electronvolt} ohne externes Magnetfeld.}\label{fig:spds}
\end{figure}

\begin{figure}[H]
    \centering
    \includegraphics{bilder/plots/Bz_0mT/autocorr_26.03meV.pdf}
    \caption{vergleich Autokorrelation, von ursprünglichem und bereinigtem Telegrafenrauschen und Differenz. Simulation bei \SI{26.04}{\milli\electronvolt} ohne externes Magnetfeld.}\label{fig:autocorr}
\end{figure}

\subsection{Variation der Temperatur}

Im Rahmen der Masterarbeit von Julius Schlegel \cite{schlegel-master} wurden bereits Simulationen für verschiedene Temperaturen durchgeführt. Diese Daten wurden für diesen Teil der Analyse verwendet

\subsubsection{In der Zeitdomäne}

\begin{figure}[H]
    \centering
    \subcaptionbox{Temperaturbereich mit sichtbarem Telegrafenrauschen \label{fig:temp-hist-tg}}{\includegraphics{bilder/plots/temp_comparison_long/mc_hist_tgr.pdf}}
    \subcaptionbox{größerer Temperaturbereich um den Reorientierungsübergang \label{fig:temp-hist-long}}{\includegraphics{bilder/plots/temp_comparison_long/mc_hist.pdf}}
    \caption{Wahrscheinlichkeitsdichteverteilung von \(m_c\) für eine Kombination von mehreren Simulationsläufen.}\label{fig:temp-hist}    
\end{figure}

\todo{Grund für unterschiedlich hohe Peaks}

\begin{figure}[H]
    \centering
    \includegraphics{bilder/plots/temp_comparison_long/mc_diff_hist.pdf}
    \caption{Wahrscheinlichkeitsdichteverteilung für Differenz zwischen rohem Signal und extrahiertem Telegrafenrauschen von \(m_c\) (entspricht extrahiertem statistisches Rauschen)}\label{fig:temp-diff-hist}    
\end{figure}

\todo{Mutmaßung warum breitere Peaks bei aktivem Telegrafenrauschen (oberhalb und unterhalb spitzer)}

\begin{figure}[H]
    \centering
    \subcaptionbox{einzelner Datenpunkt für jede einzelne Simulation \label{fig:temp-switch-count-scatter}}{\includegraphics{bilder/plots/temp_comparison/switch_count_scatter.pdf}}
    \subcaptionbox{Mittel-, Extremwerte und Verteilung für jede Simulationstemperatur \label{fig:temp-switch-count-violin}}{\includegraphics{bilder/plots/temp_comparison/switch_count_violin.pdf}}
    \caption{Temperaturabhängigkeit der Anzahl an Schaltvorgängen \todo{pro welches Zeitinterval?} \todo{über kette berechneten wert in Plot b einfügen}}\label{fig:temp-switch-count}
\end{figure}

\todo{verknüpfung mit \cref{fig:temp-hist-tg}}

\begin{figure}[H]
    \centering
    \includegraphics{bilder/plots/temp_comparison/switch_events.pdf}
    \caption{Zeitpunkt an dem ein Schaltvorgang stattfindet \todo{use ns}\todo{use scatter?}}\label{fig:switch-events}
\end{figure}

\todo{treten hier probleme aufgrund der Verkettung auf}

\todo{Ausreißer datenpunkte vs kette}

\begin{figure}[H]
    \centering
    \subcaptionbox{einzelner Datenpunkt für jede einzelne Simulation\label{fig:temp-up-percentage-scatter}}{\includegraphics{bilder/plots/temp_comparison/up_percentage_scatter.pdf}}
    \subcaptionbox{Mittel-, Extremwerte und Verteilung für jede Simulationstemperatur \label{fig:temp-up-percentage-violin}}{\includegraphics{bilder/plots/temp_comparison/up_percentage_violin.pdf}}
    \caption{Aufenthaltszeit im oberen Zustand}\label{fig:temp-up-percentage}
\end{figure}

\todo{Gleichgewicht bei 50\%, aber immer schwieriger zu betrachten, weil MDT größer wird}

\subsubsection{In der Frequenzdomäne}

\todo{Erklärrung, warum die Daten in der Zeitdomäne experimentell nicht messbar sind.}

\begin{figure}[H]
    \centering
    \subcaptionbox{source \label{fig:temp-spd-source}}{\includegraphics{bilder/plots/temp_comparison/spectral_power_density.pdf}}
    \subcaptionbox{clean \label{fig:temp-spd-clean}}{\includegraphics{bilder/plots/temp_comparison/spectral_power_density_cleaned.pdf}}
    \subcaptionbox{diff\label{fig:temp-spd-diff}}{\includegraphics{bilder/plots/temp_comparison/spectral_power_density_diff.pdf}}
    \caption{Spektrale Leistungsdichten für verschiedene Temperaturen \todo{nur eine colorbar?}}\label{fig:temp-spd}
\end{figure}


\begin{figure}[H]
    \centering
    \subcaptionbox{source \todo{mit fit?}\label{fig:temp-autocorr-source}}{\includegraphics{bilder/plots/temp_comparison/autocorrelation.pdf}}
    \subcaptionbox{clean \label{fig:temp-autocorr-clean}}{\includegraphics{bilder/plots/temp_comparison/autocorrelation_cleaned.pdf}}
    \subcaptionbox{diff\label{fig:temp-autocorr-diff}}{\includegraphics{bilder/plots/temp_comparison/autocorrelation_diff.pdf}}
    \caption{temp autocorrelations}\label{fig:temp-autocorr}
\end{figure}


\todo{Berechnung MDT aus Autokorrelation}

\begin{figure}[H]
    \centering
    \includegraphics{bilder/plots/temp_comparison/mean_dwell_time_comparison.pdf}
    \caption{Mittlere Aufenthaltszeiten aus Zeitlicher Analyse (Violinenplot) und aus exponentiellem Fit der Autokorrelationsfunktion \todo{mdt aus kette?}}\label{fig:temp-mdt-comp}
\end{figure}

\todo{underschoot problem? für niedrige temperaturen?}

\begin{figure}[H]
    \centering
    \includegraphics{bilder/plots/temp_comparison_long/mean_dwell_time.pdf}
    \caption{temp mean dwell time long\todo{auch mdt aus autocorr?} \todo{plot mit \cref{fig:temp-mdt-comp} zusammenlegen}}\label{fig:temp-mdt-long}
\end{figure}


\begin{figure}[H]
    \centering
    \includegraphics{bilder/plots/temp_comparison_long/rauschamplitude.pdf}
    \caption{temp autocorrelation amplitude (Rauschamplitude)}\label{fig:temp-autocorr-amplitude}
\end{figure}

\subsection{In externem Magnetfeld}

\todo{in plots \(B_z\) statt \(Bz\) nutzen}

\begin{figure}[H]
    \centering
    \includegraphics{bilder/plots/Bz_comparison/critical_temperature.pdf}
    \caption{Bz critical temperature}\label{fig:bz-crit-temp}
\end{figure}

\todo{erklärung bei so geringen Magnetfeldern wird der Reorientierungsübergang nicht aufgeweicht}

\todo{Plot bei 1T von Julius einfügen}

\todo{welche bedingungen Bleiben für die nachfolgenden Simulationen gleich?}

\todo{z = c}

\todo{eigenschaften B-Feld}

\todo{Einfluss Vorzeichenumkehr Bz}

\begin{figure}[H]
    \centering
    \includegraphics{bilder/plots/Bz_sign_comparison/20mT_hist_comp.pdf}
    \caption{Bz sign comparison hist}\label{fig:bz-sign-hist}
\end{figure}

\todo{Annahme, dass sich das System Symmetrisch verhällt}


\subsubsection{In der Zeitdomäne}

\begin{figure}[H]
    \centering
    \includegraphics{bilder/plots/max_Bz/mc_hist.pdf}
    \caption{Wahrscheinlichkeitsdichteverteilung von \(m_c\) für eine Kombination von mehreren Simulationsläufen und bei verschieden starken externen Magnetfeldern}\label{fig:b-hist}    
\end{figure}

\todo{Schwierigkeiten, wenn linker Peak zu klein Wird}

\todo{erklärung verschiebung der Niveau peaks}

\todo{was passiert in den anderen Komponenten}

\begin{figure}[H]
    \centering
    \includegraphics{bilder/plots/max_Bz/mc_time.pdf}
    \caption{Ausschnitt von \(m_c\) im Zeitlichen Verlauf bei verschieden starken externen Magnetfeldern . Die Colormap ist hier identisch zu der von \cref{fig:b-hist}. Die y-Achse reicht hier jeweils von \num{-3e-4} bis \num{+3e-4}}\label{fig:b-time}    
\end{figure}

\begin{figure}[H]
    \centering
    \includegraphics{bilder/plots/max_Bz/switch_events.pdf}
    \caption{Zeitpunkt an dem ein Schaltvorgang stattfindet}\label{fig:bz-switch-events}   
\end{figure}

\todo{hier bei 100mT garantiert kein problem aufgrund der Verkettung, da sich das system quasi durchgängig im oberen Niveau befindet}

In Manchen Simulationsläufen bei \SI{100}{\milli\tesla} findet gar kein Schaltvorgang statt.

\begin{figure}[H]
    \centering
    \subcaptionbox{Aufenthaltszeit im oberen Zustand. Bestimmt über einzelne Simulationsläufe (Violinplot) und über Kette mehrerer Simulationen (schwarze Punkte) \label{fig:bz-up-percentage-violin}}{\includegraphics{bilder/plots/max_Bz/up_percentage_violin.pdf}}
    \subcaptionbox{Mittlere Aufenthaltszeit in oberem (blau) und unterem (grün) Zustand in Prozent. Die wagrechte Linie kennzeichnet hierbei den Mittelwert und der Violinplot die Verteilung der verschiedenen Aufenthaltszeiten auf dem jeweiligen Niveau \label{fig:bz-state-times-comp}}{\includegraphics{bilder/plots/max_Bz/state_times_comp.pdf}}
    \caption{Abhängigkeit der Aufenthaltszeit von der Magnetfeldstärke}\label{fig:bz-state-times}
\end{figure}

\todo{mittlere Aufenthaltszeit (nicht aufgeteilt nach up/down)}

\subsubsection{In der Frequenzdomäne}


\begin{figure}[H]
    \centering
    \subcaptionbox{source\label{fig:bz-spd-source}}{\includegraphics{bilder/plots/max_Bz/spectral_power_density.pdf}}
    \subcaptionbox{clean\label{fig:bz-spd-clean}}{\includegraphics{bilder/plots/max_Bz/spectral_power_density_cleaned.pdf}}
    \subcaptionbox{diff\label{fig:bz-spd-diff}}{\includegraphics{bilder/plots/max_Bz/spectral_power_density_diff.pdf}}
    \caption{Spektrale Leistungsdichten bei verschiedenen Magnetfeldstärken}\label{fig:bz-spd}
\end{figure}


\begin{figure}[H]
    \centering
    \subcaptionbox{source \label{fig:bz-autocorr-source}}{\includegraphics{bilder/plots/max_Bz/autocov.pdf}}
    \subcaptionbox{clean \todo{ohne fit und nur bis 75ps}\label{fig:bz-autocorr-clean}}{\includegraphics{bilder/plots/max_Bz/autocov_cleaned.pdf}}
    \subcaptionbox{diff\label{fig:bz-autocorr-diff}}{\includegraphics{bilder/plots/max_Bz/autocorrelation_diff.pdf}}
    \caption{Bz autocorrelations \todo{nur eine colorbar. plots etwas größer}}\label{fig:bz-autocov}
\end{figure}

\todo{Autokorrelation vs Autokovarianz}


\todo{Die Mean Dwell time ist nicht aus der Autokorrelation Bestimmbar (Mittlere Mean Dwell time)}

\begin{figure}[H]
    \centering
    \includegraphics{bilder/plots/max_Bz/t_corr.pdf}
    \includegraphics{bilder/plots/max_Bz/amplitude_corr.pdf}
    \caption{Aus irgendwelchen gründen stimmen Die Datenpunkte nicht miteinander überein \todo{Nachvollziehen, woher die Datenpunkte kommen}\todo{versuchen nochmal neu zu implementieren}\todo{plots überhaupt einbinden?}}\label{fig:bz-autocorr-amplitude}
\end{figure}

\begin{figure}[H]
    \centering
    \includegraphics{bilder/plots/max_Bz/up_percentage_fit.pdf}
    \caption{b up percentage sigmoid fit \todo{quelle der Datenpunkte}}\label{fig:bz-up-percentage}
\end{figure}

\todo{fitparameter obere schranke diskutieren}

\todo{worin ist das sonst erkennbar, dass kein Schaltvorgang mehr stattfindet?}


\begin{figure}[H]
    \centering
    \includegraphics{bilder/plots/max_Bz/autocov_high.pdf}
    \caption{Autocovarianz für hohe Magnetfeldstärken}\label{fig:bz-autocov-high}
\end{figure})



\begin{figure}[H]
    \centering
    \includegraphics{bilder/plots/max_Bz/rauschamplitude.pdf}
    \caption{Bz rauschamplitude \todo{was bedeutet das?} \todo{daten oberhalb von 90mT einschließen?}}\label{fig:bz-rauschampl}
\end{figure}Mittlere Mean Dwell time)


% bibliography (temporary)
% \bibliography{literatur} \todo{comment out before compiling main.tex}

\end{document}