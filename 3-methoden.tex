\documentclass[main.tex]{subfiles}
\begin{document}
\newpage
\section{Simulationsmethoden und Datenanalyse}

\subsection{Numerische Simulation}
Um Das System von mehreren magnetischen Momenten zu simulieren, wird das
Programm \texttt{cuteLLG}\cite{cuteLLG} verwendet. \texttt{cuteLLG} wurde von
Andread Donges (ehemaliger Mitarbeiter der AG Nowak) entwickelt und von
verschiedenen Mitarbeitern der AG Nowak modifiziert.

\todo{Erklärung wie die numerische lösung von Differentialgleichungen
    funktioniert}

\todo{Vergleich numerischer verfahren}

Bei der Bewegungsgleichung (LLG) handelt es sich um eine Differentialgleichung,
die nicht analytisch gelöst werden kann.
Deshalb wird die Differentialgleichung numerisch gelöst. Dafür gibt es
verschiedene Verfahren, die sich in der Genauigkeit und der Laufzeit
unterscheiden.

Die einfachste Methode ist das Euler-Verfahren. Dabei wird die
Differentialgleichung durch eine lineare Näherung ersetzt. Die Genauigkeit ist
dabei von der Schrittweite abhängig. Je kleiner die Schrittweite, desto genauer
ist die Näherung. Die Laufzeit ist dabei proportional zur Anzahl der Schritte.
Der Fehler steigt also mit einer längeren Laufzeit immer weiter an. Daher ist
das Euler-Verfahren für lange Simulationen ungeeignet.

\todo{Runge-Kutta-Verfahren}

\todo{Heun-Verfahren}

\cite{Computerphysik}

\todo{Implementation von cuteLLG mit dem Heun-Verfahren}

\subsection{Algorithmus zur Extraktion von Telegrafenrauschen}

\todo{figure Gauss Peaks, die ineinander übergehen}

\todo{Vergleich mit kmeans oä. (overshoot problem)}

\todo{Erklärung eigener Abwandlung, welche weniger komplex als
    \cite{random-telegraph-analysis} ist und trotzdem das overshoot problem löst}

\subsection{Fouriertransformation}

Für die Analyse in der Frequenzdomäne wird die diskrete Fouriertransformation
benötigt. Dafür wird die \enquote{fast Fourier transformation} (FFT) verwendet,
welche die Berechnung der Fouriertransformation mit einer Laufzeit von
$\mathcal{O}(n \log n)$ ermöglicht. Die FFT wird in der Bibliothek
\texttt{numpy}\cite{numpy} implementiert.

Die diskrete Fouriertransformation einer Funktion $a_m$ mit $m = 0, \dots, n-1$
wird definiert als:
\begin{align}
    A_k = \sum_{m=0}^{n-1} a_m \exp(-2 \pi i\frac{ k m}{n}) \quad \text{mit}
    \quad k = 0, \dots, n-1
\end{align}

und die inverse Fouriertransformation als:

\begin{align}
    a_m = \frac{1}{n} \sum_{k=0}^{n-1} A_k \exp(2 \pi i\frac{ k m}{n}) \quad
    \text{mit} \quad m = 0, \dots, n-1
\end{align}\cite{numpy-fft}

% Korelationsfunktion?

% bibliography (temporary)
% \bibliography{literatur} \todo{comment out before compiling main.tex}

\end{document}