\documentclass[main.tex]{subfiles}
\begin{document}
\section{Simulationsmethoden und Datenanalyse}

\subsection{Numerische Simulation}
Um Das System von mehreren magnetischen Momenten zu simulieren, wird das programm \texttt{cuteLLG}\cite{cuteLLG} verwendet, welches von Andread Donges entwickelt wurde. Es basiert auf der \enquote{Landau-Lifshitz-Gilbert} (LLG) Gleichung, welche die Bewegung von magnetischen Momenten beschreibt. Die LLG Gleichung ist definiert als:

\subsection{Algorithmus zur Extraktion von Telegrafenrauschen}


\cite{random-telegraph-analysis}
\subsection{Fouriertransformation}

Für die Analyse in der Frequenzdomäne wird die diskrete Fouriertransformation benötigt. Dafür wird die \enquote{fast Fourier transformation} (FFT) verwendet, welche die Berechnung der Fouriertransformation mit einer Laufzeit von $\mathcal{O}(n \log n)$ ermöglicht. Die FFT wird in der Bibliothek \texttt{numpy}\cite{numpy} implementiert.

Die diskrete Fouriertransformation einer Funktion $f(t)$ ist hier definiert als:

\begin{align}
    \tilde{f}(\omega) = \int_{-\infty}^{\infty} f(t) e^{-i \omega t} \dd{t}
\end{align}

und die inverse Fouriertransformation als:

\begin{align}
    f(t) = \frac{1}{2 \pi} \int_{-\infty}^{\infty} \tilde{f}(\omega) e^{i \omega t} \dd{\omega}
\end{align}\cite{numpy-fft}

% bibliography (temporary)
\bibliography{literatur}

\end{document}

