\documentclass[main.tex]{subfiles}
\begin{document}
\section{Theoretische Grundlagen}

In diesem Abschnitt werden die theoretischen Grundlagen, die diese Arbeit
benötigt, erläutert.

\subsection{Heisenberg Modell}

Um das System von mehreren magnetischen Momenten zu beschreiben, wird das
klassische Heisenberg Modell verwendet. Es beschreibt die Wechselwirkung
zwischen den magnetischen Momenten der einzelnen Atome.

Die magnetischen Momente \(\vec{\mu}\) werden als Vektoren der Länge
\(\mu_{\text{S}}\) angenommen.
Die normierten und dimensionslosen magnetischen Momente werden mit \(\vec{S}\)
bezeichnet:

\begin{align}
	\vec{S} = \frac{\vec{\mu}}{\mu_{\text{S}}}
\end{align}

Da ein Antiferromagnet betrachtet wird, bei dem die magnetischen Momente in
entgegengesetzte Richtungen zeigen und die Beträge der magnetischen Momente
gleich sind gilt zusätzlich \(\mu_{\text{S},i} = \mu_{\text{S},j} = \mu_{S}\).

Diese Wechselwirkungen werden durch die Hamilton Funktion beschrieben, welche
sich aus mehreren Komponenten zusammensetzt:

\subsubsection*{Heisenberg Austauschwechselwirkung}
% Heisenberg Austauschwechselwirkung

Die Heisenberg Austauschwechselwirkung beschreibt die Wechselwirkung zwischen
den magnetischen Momenten der einzelnen Atome aufgrund von überlappten
Wellenfunktionen~\cite{Heisenberg-Ferromagnetismus}. Sie ist die dominierende
Wechselwirkung in Antiferromagneten. Klassisch ist sie proportional zum
Skalarprodukt der magnetischen Momente:

\begin{align}
	\mathcal{H}_{\text{EXC}} = -\sum_{i,j} J_{ij} \vec{S}_i \cdot \vec{S}_j\label{eq:hamilton-heisenberg-exc}
\end{align}

\subsubsection*{Dzyaloshinskii-Moriya Wechselwirkung}
% Dzyaloshinskii-Moriya interaction 

Die Dzyaloshinskii-Moriya Wechselwirkung (DMI) beschreibt die indirekte
antisymmetrische Interaktion von zwei Spins\cite{DMI}. Sie ist in
Antiferromagneten die zweitstärkste Wechselwirkung und ist proportional zum
Kreuzprodukt der magnetischen Momente:

\begin{align}
	\mathcal{H}_{\text{DMI}} = \sum_{i,j} \vec{D}_{ij} \cdot (\vec{S}_i
	\times
	\vec{S}_j)\label{eq:hamilton-dmi}
\end{align}

\subsubsection*{Magnetische Anisotropien}
Magnetische Momente können eine Präferenz für eine bestimmte Richtung haben.
Die Energie, die aufgewendet werden muss, um das magnetische Moment aus dieser
Richtung zu drehen, wird als Anisotropie bezeichnet. Es gibt verschiedene Arten
von Anisotropien.

Die hier betrachteten Anisotropien sind verschiedene Kristallanisotropien.

Eine Kristall-Anistotropie zweiter Ordnung trägt folgenden Term zur Hamilton Funktion bei:
\begin{align}
	\mathcal{H}_{\text{2.AI}} = -\sum_{i} \qty(\vec{S}_i)^\dagger
	\mathbf{d}_i
	\vec{S}_i\label{eq:hamilton-2a}
\end{align}

Ein weiterer Term, der zur Hamilton Funktion beiträgt, ist die kubische Anisotropie, welche eine Anisotropie vierter Ordnung ist:
\begin{align}
	\mathcal{H}_{\text{KAI}} = -\sum_{i} L_{i,x} \qty(\vec{S}_{i,y})^2
	\qty(\vec{S}_{i,z})^2 + L_{i,y} \qty(\vec{S}_{i,z})^2
	\qty(\vec{S}_{i,x})^2
	+
	L_{i,z} \qty(\vec{S}_{i,x})^2 \qty(\vec{S}_{i,y})^2\label{eq:hamilton-4a}
\end{align}\cite{GrossMarx}

Außerdem wird noch die Zwei-Ionen Anisotropie berücksichtigt:
\begin{align}
	\mathcal{H}_{\text{TIA}} = -\sum_{i,j} \vec{S}_i \mathbf{\kappa}_{ij}
	\vec{S}_j\label{eq:hamilton-tia}
\end{align}
\subsubsection*{Zeeman Wechselwirkung}
% Zeeman
Bei einem äußeren Magnetfeld \(B_\text{ext}\) wirkt auf jedes magnetische
Moment auch noch die Zeeman Wechselwirkung:

\begin{align}
	\mathcal{H}_{\text{Z}} = - B_\text{ext} \sum_{i} \vec{\mu}_i = -
	B_\text{ext} \sum_{i} \mu_{\text{S},i} \cdot \vec{S}_i\label{eq:hamilton-zeeman}
\end{align}

% Dipol Dipol wechselwirkung
\subsubsection*{Dipol-Dipol Wechselwirkung}

Jedes magnetische Moment erzeugt ein Magnetfeld, welches aus hinreichender
Entfernung stehts wie ein Dipolfeld aussieht:

\begin{align}
	\vec{B}(\vec{r}) & = \frac{\mu_0}{4\pi} \qty[3 \frac{\vec{r}(\vec{\mu}
			\cdot
			\vec{r})}{r^5} -
		\frac{\vec{\mu}}{r^3}]\label{eq:dipolmoment}
\end{align}\cite{Nolting-3-elektrodynamik}

Wenn zwei magnetische Momente \(\vec{\mu}_i\) und \(\vec{\mu}_j\) in einem
Abstand \(r_{ij}\) zueinander stehen, wirkt auf das magnetische Moment
\(\vec{\mu}_i\) das Magnetfeld \(\vec{B}_j\) und auf das magnetische Moment
\(\vec{\mu}_j\) das Magnetfeld \(\vec{B}_i\). Die Wechselwirkungsenergie
zwischen den beiden magnetischen Momenten ist die Energie, die aufgewendet
werden muss, um die beiden magnetischen Momente aus dem Magnetfeld des jeweils
anderen zu entfernen. Die Wechselwirkungsenergie ist also proportional zum
Skalarprodukt der beiden Magnetfelder:

\begin{align}
	\mathcal{H}_{\text{DD}} & = \sum_{\substack{i,j           \\ i \neq j}}
	\frac{\mu_0}{4\pi}
	\cdot  \frac{3(\vec{\mu}_i \cdot \vec{r}_{ij})(\vec{\mu}_j \cdot
		\vec{r}_{ij})
	- \vec{\mu}_i \cdot \vec{\mu}_j \cdot r_{ij}^2}{r_{ij}^5} \\
	                        & = \sum_{\substack{i,j           \\ i \neq j}}
	\frac{\mu_0 \cdot \mu_{\text{S,i}} \cdot \mu_{\text{S,j}}}{4\pi}
	\cdot \frac{3(\vec{S}_i \cdot \vec{r}_{ij})(\vec{S}_j \cdot
		\vec{r}_{ij})
		- \vec{S}_i \cdot \vec{S}_j \cdot r_{ij}^2}{r_{ij}^5}\label{eq:hamilton-dd}
\end{align}

\subsubsection*{Gesamte Hamilton Funktion}

Die Gesamte Hamilton Funktion setzt sich aus den einzelnen Komponenten zusammen:

\begin{align}
	\mathcal{H} & = \mathcal{H}_{\text{EXC}} + \mathcal{H}_{\text{DMI}} +
	\mathcal{H}_{\text{2.AI}} + \mathcal{H}_{\text{KAI}} +
	\mathcal{H}_{\text{TIA}} + \mathcal{H}_{\text{Z}} +
	\mathcal{H}_{\text{DD}}\label{eq:hamilton}\\
	&= \todo{einsetzen}
\end{align}

\subsection{Bewegungsgleichung: Landau-Lifschitz-Gilbert Gleichung}
% Landau-Lifschitz Gleichung (1935)
Um die Bewegung der magnetischen Momente zu beschreiben, muss aus der
Hamiltonfunktion die Bewegungsgleichung abgeleitet werden.

Obwohl die Hamiltonfunktion klassisch ist, kann die Bewegungsgleichung über die
Quantenmechanik hergeleitet werden. Das Ehrenfest-Theorem besagt, dass die
zeitliche Änderung des Erwartungswertes eines Operators gleich dem
Erwartungswert des Kommutators des Operators mit der Hamiltonfunktion ist:

\begin{align}
	\dv{\hat{\vec{S}}_i}{t} = \frac{i}{\hbar} \expval{\qty[\hat{\vec{S}}_i,
			\hat{\mathcal{H}}]}
\end{align}\cite{qm-1-Schwabl}

Mit der Relation \(\hat{\vec{S}}_i = \frac{\gamma_i}{\mu_{\text{S}}}
\hat{\vec{J}}_i\) und der Kommutatorrelation \(\qty[\hat{\vec{J}}_i,
	\hat{\vec{J}}_j] = i \hbar \cdot \varepsilon_{abc}
\hat{\vec{J}}_{c,i}\) (wobei
\(\varepsilon\) das Levi-Civita Symbol ist) ergibt sich:

\begin{align}
	\dv{\vec{S}_i}{t}                             & = -\frac{\gamma
	}{\mu_{\text{S}}}
	\qty(\vec{S}_i \times
	\vec{\mathcal{H}}_{\text{i}})\label{eq:landau-lifschitz}
	\\
	\text{mit} \quad \vec{\mathcal{H}}_{\text{i}} & = -
	\pdv{\mathcal{H}}{\vec{S}_i}
\end{align}\cite{landau-lifshitz}

% LL + Gilbert Gleichung (1955)
Diese Gleichung wurde 20 Jahre später von Gilbert um einen Dämpfungsterm
erweitert:

\begin{align}
	\dv{\vec{S}_i}{t} & = -\frac{\gamma }{\mu_{\text{S}}}\qty(\vec{S}_i
	\times
	\vec{\mathcal{H}}_{\text{i}}) + \alpha \qty(\vec{S}_i \times
	\dv{\vec{S}_i}{t})
\end{align}\cite{Gilbert-damping}

Um davon auf eine klassische explizite Bewegungsgleichung zu kommen, muss die Gleichung zuerst in sich selbst eingesetzt werden:
\begin{align}
	\dv{\vec{S}_i}{t} & = -\frac{\gamma }{\mu_{\text{S}}}\qty(\vec{S}_i
	\times
	\vec{\mathcal{H}}_{\text{i}}) + \alpha \qty(\vec{S}_i \times
	\qty(-\frac{\gamma }{\mu_{\text{S}}}\qty(\vec{S}_i
		\times
		\vec{\mathcal{H}}_{\text{i}}) + \alpha \qty(\vec{S}_i \times
	\dv{\vec{S}_i}{t})))                                                \\
	                  & = -\frac{\gamma }{\mu_{\text{S}}}\qty(\vec{S}_i
	\times
	\vec{\mathcal{H}}_{\text{i}}) - \frac{\alpha\gamma}{\mu_\text{S}}
	\qty(\vec{S}_i \times \qty(\vec{S}_i \times
		\vec{\mathcal{H}}_{\text{i}})) +
	\alpha^2 \qty(\vec{S}_i \times \qty(\vec{S}_i \times
		\dv{\vec{S}_i}{t}))
\end{align}
Aufgrund der Identitäten \(\vec{a} \times \qty(\vec{b} \times \vec{c}) = \vec{b} \qty(\vec{a} \cdot \vec{c}) - \vec{c} \qty(\vec{a} \cdot \vec{b})\), \(\vec{S}_i \bot \dv{\vec{S}_i}{t}\) und \(\vec{S}_i \cdot \vec{S}_i = 1\) ergibt sich:
\begin{align}
	\alpha^2 \qty(\vec{S}_i \times \qty(\vec{S}_i \times
	\dv{\vec{S}_i}{t})) & = -\alpha^2\dv{\vec{S}_i}{t}
\end{align}
Somit lässt sich die Bewegungsgleichung nach \(\dv{\vec{S}_i}{t}\) auflösen:

\begin{align}
	\dv{\vec{S}_i}{t} & = -\frac{\gamma
	}{\mu_{\text{S}}(1+\alpha^2)}\qty[\underbrace{\vec{S}_i \times \vec{\mathcal{H}}_{\text{i}}}_\text{Präzession} + \underbrace{\alpha \vec{S}_i \times
		\qty(\vec{S}_i \times
			\vec{\mathcal{H}}_{\text{i}})}_\text{Dämpfung}]\label{eq:llg}
\end{align}
% \todo{erklärung der einzelnen Terme}
Beim Vergleich mit der Landau-Lifschitz-Gleichung \eqref{eq:landau-lifschitz} fällt auf, dass sich einerseits der vorfaktor geändert hat und andererseits ein Dämpfungsterm hinzugekommen ist. Für \(\alpha = 0\) sind die Gleichungen identisch. 

Der Präzessionsterm beschreibt die Präzession der magnetischen Momente um das effektive Feld.

Der Dämpfungsterm beschreibt die Energie, die aufgewendet werden muss, um die magnetischen Momente zu drehen. Der Dämpfungsterm ist also eine Art Reibungsterm, der die Präzession der magnetischen Momente dämpft.

Was jetzt noch fehlt ist das weiße Rauschen von \( \vec{\mathcal{H}}_i \), welches durch die endliche Temperatur entsteht. Dieses wird durch einen stochastischen Term \(\vec{\xi}_i(t)\) beschrieben, der die Bewegungsgleichung zu einer stochastischen Differentialgleichung macht:

\begin{align}
	\vec{\mathcal{H}}_{\text{i}}                 & =
	-\pdv{\mathcal{H}}{\vec{S}_i} +
	\vec{\xi}_i(t)
	\\
	\expval{\vec{\xi}_i(t)}                      & = 0
	\\
	\expval{\xi_{i,\alpha}(t) \xi_{j,\beta}(t')} & = \frac{2\mu_\text{S}
		\alpha
	}{\gamma}k_\text{B} T \cdot \delta_{ij} \delta_{\alpha \beta}
	\delta(t-t')
\end{align}

\begin{figure}[H]
	\centering
	\subcaptionbox{\(T = \SI{0}{\kelvin} \)}
	{\includegraphics[width=0.45\textwidth]{bilder/jschlege/LLG_T0_labeled.png}}
	\subcaptionbox{\(T \gg \SI{0}{\kelvin} \)}
	{\includegraphics[width=0.45\textwidth]{bilder/jschlege/LLG_labeled.png}}
	\caption{Simulation der LLG-Gleichung. Der blaue Vektor entspricht dem Spin \(\vec{S}\) und die blaue Linie ist der Pfad, den die Pfeilspitze zurücklegt. Der Graue Vektor entspricht dem effektiven Feld, ohne Rauschen. Der Ptäzessionsterm wird durch den roten vektor beschrieben und der Dämpfungsterm durch den grünen. \cite{schlegel-master}}
	\label{fig:llg-rauschen}
\end{figure}

\subsection{Modellierung von Orthoferrit}

Orthoferrite sind chemische Verbindungen der Form \ce{RFeO3}, wobei \ce{R} ein
Rest ist, durch den sich die verschiedenen Orthoferrite unterscheiden. Hier
bildet Erbium (\ce{Er}) diese Restgruppe.

% \begin{figure}[htbp]
% 	\centering
% 	\includegraphics[width=0.4\textwidth]{bilder/jschlege/UnitCell_labeled.png}
% 	\caption{Kristallstruktur von
% 		\ce{ErFeO3}. Die Eisenatome sind dabei blau, die
% 		Sauerstoffatome rot und die Erbiumatome magenta gefärbt.
% 		\cite{schlegel-master}}
% 	\label{fig:orthoferrit}
% \end{figure}

% \todo{erklärung für Samarium (reorientierungstemperatur)}

Die Reorientierungstemperatur \(T_l < T < T_h\) liegt bei Erbiumferrit bei ungefähr \SI{110}{\kelvin}\cite{Deng2015}. Durch teilweises Ersetzen von Erbium durch Samarium kann die Reorientierungstemperatur erhöht werden. Bei (\ce{Sm_{0.7}Er_{0.3}FeO3}) liegt sie ungefähr bei Raumtemperatur, weshalb es im Experiment einfacher zu untersuchen ist.

% Parameter für Simulation
% \todo{wer hat die Parameter bestimmt?}

Für die Simulation wird ein Modell von \ce{ErFeO3} verwendet, bei dem die Anisotropie-Parameter angepasst wurden, um eine Reorientierungstemperatur bei Raumtemperatur wie bei \ce{Sm_{0.7}Er_{0.3}FeO3} zu erhalten:

\begin{align}
	J_1             & = \SI{-22,32}{\milli\electronvolt}, \quad J_2 =
	\SI{-1,4}{\milli\electronvolt}\label{eq:params-exc}
	\\
	\mathbf{d}      & = \mqty(
	\num{0.0153}    & \num{-0.0153}                                   & 0
	\\
	\num{-0.0153}   & \num{0.0153}                                    & 0
	\\
	0               & 0                                               &
	\num{0.905})\label{eq:params-2a}
	\\
	L_x             & = L_y = L_z = \SI{0.036}{\milli\electronvolt}\label{eq:params-4a}
	\\
	\mathbf{\kappa} & = \mqty(
	0               & 0                                               & 0
	\\
	0               & 0                                               & 0
	\\
	0               & 0                                               &
	\num{-0.1255})\label{eq:params-tia}
	\\
	\vec{D}_x       & = \mqty*(\num{\pm 0.0036}
	\\ \num{\pm 0.1268} \\ \num{\pm
		0.1255}), \quad
	\vec{D}_y = \mqty*(\num{\pm 0.1268}
	\\ \num{\pm 0.0036} \\ \num{\pm
		0.1255}), \quad
	\vec{D}_z = \mqty*(\num{\pm 0.1038}
	\\ \num{\pm 0.2252} \\ 0)\label{eq:params-dmi}
\end{align}

\(J_1\) ist hier der Parameter für die Heisenberg-Austauschwechselwirkung \eqref{eq:hamilton-heisenberg-exc} mit den nächsten Nachbarn (6) und \(J_2\) die mit den übernächsten Nachbarn (12). \(\mathbf{d}\) beschreibt die uniaxiale Anisotropie\eqref{eq:hamilton-2a}, \(L_x\), \(L_y\) und \(L_z\) sind die Koeffizienten der kubischen Anisotropie\eqref{eq:hamilton-4a}, \(\mathbf{\kappa}\) ist die Zwei-Ionen Anisotropie\eqref{eq:hamilton-tia} und \(\vec{D}_x\), \(\vec{D}_y\) und \(\vec{D}_z\) sind die Dzyaloshinskii-Moriya Vektoren\eqref{eq:hamilton-dmi}.

Die Dipol-Dipol Wechselwirkung\eqref{eq:hamilton-dd} wird in der Simulation nicht berücksichtigt, da sie im Vergleich zu den anderen Wechselwirkungen sehr klein ist.

\begin{figure}[htbp]
	\centering
	\subcaptionbox{Einheitszelle mit DMI
		Vektoren(gelb)\label{subfig:orthoferrit-dmi}}
	{\includegraphics[width=0.3\textwidth]{bilder/jschlege/UnitCell_withDMI_labeled.png}}
	\subcaptionbox{Einheitszelle mit magnetischen Momenten (grün) für
		\(T<T_l\)\label{subfig:orthoferrit-below-rt}}
	{\includegraphics[width=0.3\textwidth]{bilder/jschlege/UnitCell_belowRT.png}}
	\subcaptionbox{Einheitszelle mit magnetischen Momenten (grün) für
		\(T>T_h\)\label{subfig:orthoferrit-above-rt}}
	{\includegraphics[width=0.3\textwidth]{bilder/jschlege/UnitCell_aboveRT.png}}
	\caption{Kristallstruktur von \ce{ErFeO3}  Die Eisenatome sind dabei blau, die Sauerstoffatome rot und die Erbiumatome magenta gefärbt.
	\textbf{(a)} Die DMI Vektoren sind gelb eingezeichnet und die Richtung entspricht den Werten in \cref{eq:params-dmi} 
	\textbf{(b)} Unterhalb der Reorientierungstemperatur Dominiert die Anisotropie zweiter Ordnung (\cref{eq:hamilton-2a,eq:params-2a}), weshalb die magnetischen Momente quasi Parallel zur c-Achse ausgerichtet sind.
	\textbf{(c)} Oberhalb der Reorientierungstemperatur dominiert die Zwei-Ionen Anisotropie (\cref{eq:hamilton-tia,eq:params-tia}), weshalb die magnetischen Momente quasi Senkrecht zur c-Achse ausgerichtet sind. Die Anisotropie zweiter Ordnung sorgt dafür, dass die magnetischen Momente Quasi parallel zur a-Achse ausgerichtet sind.\cite{schlegel-master}}
	\label{fig:orthoferrit}
\end{figure}

\subsection{Telegrafenrauschen}

Das Telegrafenrauschen ist ein stochastischer Prozess, der aus zwei Zuständen besteht, die zufällig zwischen einander wechseln. Ein gängiges Modell ist der Dichtomische Markow-Prozess (DMP).

Dieser entspricht einer Markow-Kette mit zwei Zuständen (\(c_1\) und \(c_2\))
zwischen denen die Zustandsvariable \(X(t)\) zufällig hin und her wechselt. Aus
den Übergangsraten \(\lambda_1\) und \(\lambda_2\) bildet sich die
Übergangsmatrix \(\mathbf{W}\):

\begin{align}
	\mathbf{W} = \mqty(
	-\lambda_1 & \lambda_2   \\
	\lambda_1  & -\lambda_2)
\end{align}

Die Übergangsraten sind umgekehrt proportional zu den mittleren Verweilzeiten
(die mittlere Zeit, die \(X(t)\) in einem Zustand verbringt): \(\lambda_i =
\frac{1}{\tau_i}\).

\begin{align}
	\lambda &= \frac{1}{2} \qty(\lambda_1 + \lambda_2)\\
	e^{\mathbf{W}t} &= \frac{1}{2\lambda} \mqty(\lambda_2 + \lambda_1 e^{-2\lambda t} & \lambda_2 \qty(1- e^{-2\lambda t})\\
	\lambda_1 \qty(1- e^{-2\lambda t}) & \lambda_1 + \lambda_2 e^{-2\lambda t})
\end{align}

\begin{align}
	P(1)                            & = \frac{\tau_1}{\tau_1 + \tau_2} =
	\frac{\lambda_2 }{\lambda_1 +
		\lambda_2}
	\\
	P(2)                            & = \frac{\tau_2}{\tau_1 + \tau_2} =
	\frac{\lambda_1 }{\lambda_1 +
		\lambda_2}
	\\
	\expval{X}                      & = c_1 P(1) + c_2 P(2) = \frac{\tau_1
		c_1 + \tau_2
		c_2}{\tau_1 + \tau_2} = \frac{\lambda_2 c_1 + \lambda_1
		c_2}{\lambda_1 +
		\lambda_2}
	\\
	\expval{\qty(X - \expval{X})^2} & = \frac{\tau_1 \tau_2
		(c_1-c_2)^2}{(\tau_1 + \tau_2)^2} = \frac{\lambda_1 \lambda_2
		(c_1-c_2)^2}{(\lambda_1 + \lambda_2)^2}
\end{align}

\begin{align}
	\delta X(t) & = X(t) - \expval{X}\\
	\expval{\delta X(0)\delta X(t)} &= \frac{\lambda_1 \lambda_2
	(c_1-c_2)^2}{(\lambda_1 + \lambda_2)^2} \cdot \exp(-\lambda t)\\
	\tau_\mathrm{corr} &= \frac{1}{2\lambda} = \frac{1}{\lambda_1 +
		\lambda_2} = \frac{\tau_1 \tau_2}{\tau_1 + \tau_2}
\end{align}

\subsubsection*{Symmetrischer DMP}

Wenn \(c_1=-c_2=c\) und \(\lambda_1 = \lambda_2 = \lambda\) gilt, ist der DMP
symmetrisch und viele Ausdrücke vereinfachen sich:
\begin{align}
	\expval{X}        & = 0                    \\
	\expval{X^2}      & = c^2                  \\
	\expval{X(0)X(t)} & = c^2 \exp(-\lambda t)
\end{align}\cite{matphys}
\todo{\(e^{-2\lambda t}\) oder \(e^{-\lambda t}\)?} 
\subsubsection*{Spektrale Leistungsdichte und Autokovarianz}

Um das Rauschen in der Frequenzdomäne zu betrachten, wird die spektrale
Leistungsdichte \(S(f)\) verwendet. Sie lässt sich aus der abgeschnittenen
Fouriertransformierten der Magnetisierung berechnen:
\begin{align}
	\hat{m}_\beta^T(\omega) & = \int_0^T \dif t \ m_\beta(t) \exp(-i \omega
	t)
	\\
	P_{\beta}(\omega)       & = \lim_{T \to \infty} \frac{1}{T}
	\abs{\hat{m}_\beta^T(\omega)}^2
\end{align}

Die Spektrale Leistungsdichte lässt sich auch aus der Autokovarianz berechnen:
\begin{align}
	P_\beta(\omega) = \int_{-\infty}^{\infty} \dif \tau \ \exp(-i \omega
	\tau) \expval{m_\beta(0) m_\beta(\tau)}
\end{align}

Eine Herleitung dazu befindet sich in \cite{schlegel-master}.

Die Spektrale Leistungsdichte für den symmetrischen DMP lässt sich somit berechnen:


\begin{align}
	P_\text{DMP}(\omega) &= \int_{-\infty}^{\infty} \dif \tau \ e^{-i \omega \tau} \expval{X(0)X(\tau)}\\
	&= c^2 \int_{-\infty}^{\infty} \dif \tau \ e^{-i \omega \tau}e^{-\lambda \abs{\tau}}\\
	&= c^2 \int_{-\infty}^{0} \dif \tau \ e^{-i \omega \tau}e^{\lambda \tau} + c^2 \int_{0}^{\infty} \dif \tau \ e^{-i \omega \tau}e^{-\lambda \tau}\\
	&= c^2 \int_{0}^{\infty} \dif \tau \ e^{-\lambda \tau} \qty(e^{i \omega \tau} + e^{-i \omega \tau})\\
	&= c^2 \qty[\frac{e^{-\tau(\lambda + i\omega)}}{\lambda + i\omega} + \frac{e^{-\tau(\lambda + i\omega)}\cdot e^{\tau 2 i \omega)}}{\lambda - i\omega}]_{0}^{\infty}\\
	&= c^2\qty(\frac{1}{\lambda + i\omega} + \frac{1}{\lambda - i\omega})\\
	&= \frac{c^2 \cdot 2\lambda}{\lambda^2 + \omega^2}
\end{align}
Was einer Lorenzverteilung mit dem Maximum bei \(\omega=0\) entspricht.

% bibliography (temporary)
\bibliography{literatur} \todo{comment out before compiling main.tex}

\end{document}