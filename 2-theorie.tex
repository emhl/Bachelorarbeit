\documentclass[main.tex]{subfiles}
\begin{document}
\section{Theoretische Grundlagen}

In diesem Abschnitt werden die theoretischen Grundlagen, die diese Arbeit
benötigt, erläutert.

\subsection{Heisenberg Modell}

Um das System von mehreren magnetischen Momenten zu beschreiben, wird das
klassische Heisenberg Modell verwendet. Es beschreibt die Wechselwirkung
zwischen den magnetischen Momenten der einzelnen Atome.

Die magnetischen Momente \(\vec{\mu}\) werden als Vektoren der Länge
\(\mu_{\text{S}}\) angenommen.
Die normierten und dimensionslosen magnetischen Momente werden mit \(\vec{S}\)
bezeichnet:

\begin{align}
	\vec{S} = \frac{\vec{\mu}}{\mu_{\text{S}}}
\end{align}

Da ein Antiferromagnet betrachtet wird, bei dem die magnetischen Momente in
entgegengesetzte Richtungen zeigen und die Beträge der magnetischen Momente
gleich sind gilt zusätzlich \(\mu_{S,i} = \mu_{S,j} = \mu_{S}\).

Diese Wechselwirkungen werden durch die Hamilton Funktion beschrieben, welche
sich aus mehreren Komponenten zusammensetzt:

\subsubsection*{Heisenberg Austauschwechselwirkung}

\begin{align}
	\mathcal{H}_{\text{EXC}} = -\sum_{i,j} J_{ij} \vec{S}_i \cdot \vec{S}_j
\end{align}

\subsubsection*{Dzyaloshinskii-Moriya Wechselwirkung}
% Dzyaloshinskii-Moriya interaction 

\begin{align}
	\mathcal{H}_{\text{DMI}} = \sum_{i,j} \vec{D}_{ij} \cdot (\vec{S}_i \times
	\vec{S}_j)
\end{align}

\subsubsection*{Anisotropien}
% uniaxial anisotropy
\begin{align}
	\mathcal{H}_{\text{UA}} = -\sum_{i} \qty(\vec{S}_i)^\dagger \mathbf{d}_i
	\vec{S}_i
\end{align}

% cubic anisotropy

\begin{align}
	\mathcal{H}_{\text{CA}} = -\sum_{i} L_{i,x} \qty(\vec{S}_{i,y})^2
	\qty(\vec{S}_{i,z})^2 - L_{i,y} \qty(\vec{S}_{i,z})^2 \qty(\vec{S}_{i,x})^2
	-
	L_{i,z} \qty(\vec{S}_{i,x})^2 \qty(\vec{S}_{i,y})^2
\end{align}

% zwei ionen anisotropie

\begin{align}
	\mathcal{H}_{\text{TIA}} = -\sum_{i,j} \vec{S}_i \mathbf{\kappa}_{ij}
	\vec{S}_j
\end{align}

\subsubsection*{Zeeman Wechselwirkung}
% Zeeman

\begin{align}
	\mathcal{H}_{\text{Z}} = -\sum_{i} \vec{S}_i \cdot \mathbf{H} \cdot
	\vec{S}_i
\end{align}

% Dipol Dipol wechselwirkung
\subsubsection*{Dipol-Dipol Wechselwirkung}

\begin{align}
	\vec{B}(\vec{r}) & = \frac{\mu_0}{4\pi} \qty[3 \frac{\vec{r}(\vec{\mu}
			\cdot
			\vec{r})}{r^5} - \frac{\vec{\mu}}{r^3}]
\end{align}

\begin{align}
	\mathcal{H}_{\text{DD}} & = \sum_{\substack{i,j \\ i \neq j}}
	\frac{\mu_0}{4\pi}
	\cdot  \frac{3(\vec{\mu}_i \cdot \vec{r}_{ij})(\vec{\mu}_j \cdot
		\vec{r}_{ij})
	- \vec{\mu}_i \cdot \vec{\mu}_j}{r_{ij}^3}      \\
	                        & = \sum_{\substack{i,j \\ i \neq j}}
	\frac{\mu_0 \cdot \mu_{\text{S,i}} \cdot \mu_{\text{S,j}}}{4\pi}
	\cdot \frac{3(\vec{S}_i \cdot \vec{r}_{ij})(\vec{S}_j \cdot
		\vec{r}_{ij})
		- \vec{S}_i \cdot \vec{S}_j}{r_{ij}^3}
\end{align}

\subsection{Bewegungsgleichung: Landau-Lifschitz-Gilbert Gleichung}

% Landau-Lifschitz Gleichung (1935)
\begin{align} 
    \pdv{\vec{S}_i}{t} &= -\frac{\gamma \vec{S}_i}{\mu_{\text{S}}} \times \qty(\mathcal{H}_{\text{i}} + \alpha \vec{S}_i \times \mathcal{H}_{\text{i}}) \\
    \mathcal{H}_{\text{i}} &= - \pdv{\mathcal{H}}{\vec{S}_i}
    %&= - \pdv{\qty(\mathcal{H}_{\text{DD}} + \mathcal{H}_{\text{EXC}} + \mathcal{H}_{\text{ANI}} + \mathcal{H}_{\text{Z}}) }{\vec{S}_i} 
\end{align}

% LL + Gilbert Gleichung (1955)

\begin{align}
    \pdv{\vec{S}_i}{t} = -\frac{\gamma \vec{S}_i}{\mu_{\text{S}}(1+\alpha^2)} \times
    \qty(\mathcal{H}_{\text{i}} + \alpha \vec{S}_i \times \mathcal{H}_{\text{i}})
\end{align}

\subsection{Modellierung von Orthoferrit}

\subsection{Telegrafenrauschen}

\subsubsection*{Markow-Prozess}

% symmetrisch


% asymmetrisch



\subsubsection*{Spektrale Leistungsdichte und Autokovarianz}

% bibliography (temporary)
% \bibliography{literatur}

\end{document}