\documentclass[main.tex]{subfiles}
\begin{document}
\newpage
\section{Fazit}

\todo{Ergebnisse zusammenfassen}\\
Ohne B-Feld
\begin{itemize}
    \item Das Telegrafenrauschen ist sowohl in Zeitlicher, als auch in Frequenzauflösung Dominant (Autocorr und Spektrale leistungsdichte)
    \item die Aufenthaltspeaks lassen sich weder mit gauß, noch mit L peaks richtig fitten (ränder das eine zwischenbereich das andere)
    \item Die Aufenthaltszeiten steigen Exponentiell an (mit der temperatur)
    \item Die Häufigkeit von Switching Events sinkt mit erköhung der Temperatur
    \item Rauschamplituden (Bedeutung?)
\end{itemize}

Mit B-Feld
\begin{itemize}
    \item Temperaturbereich in dem Telegrafenrauschen stattfindet Wird von B-Feld nicht signifikant Beeinflusst (vgl Masterarbeit mit 1 T)
    \item Das System ist symmetrisch bezüglich der Polung des B-Felds (Weitere SImulation)
    \item Bis 50 mT ist das Telegrafenrauschen als solches gut beobachtbar. Bis 100mT kann das ganze Vereinzelt beobachtet werden
    \item Die Aufenthaltswahrscheinlichkeit folgt einer Sigmoid Verteilung
    \item Die Häufigkeit von Switching Events ist unverändert durch Die Stärke Des B-Felds (sofern stabiles Telegrafenrauschen Stattfindet)
    \item Die Aufenthaltszeiten auf dem Begünstigten niveau werden länger (und die auf dem unbegünstigten kürzer), im mittelwert mit der Gleichen Rate
    \item Die Mean Dwell time ist nicht aus der Autokorrelation Bestimmbar (Mittlere Mean Dwell time)
\end{itemize}
\todo{Frustrierend, wie lange es dauert, bis numerische ergebnisse da sind}\\
\todo{Ausblick auf weitere Untersuchungen und Verbesserungen}
\begin{itemize}
    \item Algorithmus mit 3 statt 2 Zuständen
    \item mean dwell time aus Frequenzauflösung (war mit b-feld nicht möglich)
    \item experimentelle reproduktion 
    \begin{itemize}
        \item Telegrafenrauschen hört bei ca. 100mT auf
        \item verhalten der raushcamplitude
    \end{itemize}
\end{itemize}

\end{document}