\documentclass[main.tex]{subfiles}
\begin{document}
\newpage
\section{Fazit}

\textbf{entwicklung algo zur extraktion tgr}\\

\textbf{Ergebnisse zusammenfassen}\\
Ohne B-Feld
\begin{itemize}
    \item Das Telegrafenrauschen ist sowohl in Zeitlicher, als auch in Frequenzauflösung Dominant (Autocorr und Spektrale leistungsdichte)
    \item Die Aufenthaltszeiten steigen Exponentiell an (mit der temperatur)
    \item Die Häufigkeit von Switching Events sinkt mit erköhung der Temperatur
\end{itemize}

Mit B-Feld
\begin{itemize}
    \item Temperaturbereich in dem Telegrafenrauschen stattfindet Wird von B-Feld nicht signifikant Beeinflusst (keine Aufweichung wie bei 1T)
    \item Das System ist symmetrisch bezüglich der Polung des B-Felds 
    \item Die Aufenthaltswahrscheinlichkeit folgt einer Sigmoid Kurve
    \item Die Aufenthaltszeiten auf dem Begünstigten niveau werden länger (und die auf dem unbegünstigten kürzer), im mittelwert mit der Gleichen Rate
    \item so dass Die Rate von Zustandswechseln und die mittlere Aufenthaltszeit sind unverändert durch Die Stärke Des B-Felds (sofern stabiles Telegrafenrauschen Stattfindet)
\end{itemize}

\textbf{Frustrierend, wie lange es dauert, bis numerische ergebnisse da sind}\\
\textbf{Ausblick auf weitere Untersuchungen und Verbesserungen}
\begin{itemize}
    \item Algorithmus mit 3 statt 2 Zuständen
    \item mean dwell time aus Frequenzauflösung (war mit b-feld nicht möglich)
    \item experimentelle reproduktion 
    \begin{itemize}
        \item Telegrafenrauschen hört bei ca. 100mT auf
        \item verhalten der raushcamplitude
    \end{itemize}
\end{itemize}

\end{document}