\documentclass[main.tex]{subfiles}
\begin{document}
\section{Einleitung}

% Der Nobelpreis für Physik wurde dieses Jahr für die experimentelle Erzeugung von Attosekunden-Lichtimpulse vergeben \cite{nobel-phy-2023}. Diese ultrakurzen Lichtimpulse haben es in den letzten Jahren ermöglicht, diverse elektromagnetische Phänomene auf sehr kurzen Zeitskalen zu untersuchen. Ein solches Phänomen ist das Spin-Rauschen in Orthoferrit, das in dieser Arbeit mittels numerischer Simulationen untersucht wird.

\begin{itemize}
    \item Paper, Masterarbeit haben sich mit dem Spin-Rauschen in Orthoferrit beschäftigt \cite{weiss-ultrafast,schlegel-master} im picosekunden-Bereich (schneller als bisher sonst)
    \item paper erscheint gleichzeitig mit dieser Arbeit.
    \item SFB 1432: \enquote{Fluctuations and Nonlinearities in Classical and Quantum Matter beyond Equilibrium} \cite{sfb-1432-b06} beschäftigt sich mit Rausch effekten 
    \item Industrie Rauscheffekte zu entfernen vgl. \cite{schlegel-bachelor, schlegel-master}
\end{itemize}

In vielen Messungen ist ein Rauschen in den Messdaten ein unbequemer Begleiter, der immer dabei ist und bei der Auswertung von den \enquote{richtigen Daten} entfernt wird. Hier ist es anders: Das Spin-Rauschen ist das interessante Signal, das es zu untersuchen gilt. Denn nahe der Reorientierungstemperatur des schwachen Antiferromagneten kann ein Telegrafenrauschen der Magnetisierung beobachtet werden \cite{weiss-ultrafast}.


\textbf{Ziele der Arbeit:}
\begin{itemize}
    \item genauere theoretische Untersuchung des Spin-Rauschens in Orthoferrit anhand der Daten aus \cite{schlegel-master}
    \item einfluss von B-Feld auf Rauschen
\end{itemize}

% bibliography (temporary)
% \bibliography{literatur} \todo{comment out before compiling main.tex}

\end{document}
