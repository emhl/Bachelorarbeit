\documentclass[main.tex]{subfiles}
\begin{document}
\section{Einleitung}

Der Nobelpreis für Physik wurde dieses Jahr für \enquote{die experimentelle Methoden, die Attosekunden-Lichtimpulse zur Untersuchung der Dynamik von Elektronen in Materie erzeugen} vergeben \cite{nobel-phy-2023}. Diese ultrakurzen Lichtimpulse haben es in den letzten Jahren ermöglicht, diverse elektromagnetische Phänomene auf sehr kurzen Zeitskalen zu untersuchen. Ein solches Phänomen ist das Spin-Rauschen in Orthoferrit, das in dieser Arbeit mittels numerischer Simulationen untersucht wird.

In vielen Messungen ist ein Rauschen in den Messdaten ein unbequemer Begleiter, der immer dabei ist und bei der Auswertung von den \enquote{richtigen Daten} entfernt wird. Hier ist es anders: Das Spin-Rauschen ist das interessante Signal, das es zu untersuchen gilt. Denn nahe der Reorientierungstemperatur des schwachen Antiferromagneten kann ein Telegrafenrauschen der Magnetisierung beobachtet werden \cite{weiss-ultrafast}.

% bibliography (temporary)
% \bibliography{literatur} \todo{comment out before compiling main.tex}

\end{document}
