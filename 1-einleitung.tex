\documentclass[main.tex]{subfiles}
\begin{document}
\section{Einleitung}

% Der Nobelpreis für Physik wurde dieses Jahr für die experimentelle Erzeugung von Attosekunden-Lichtimpulse vergeben \cite{nobel-phy-2023}. Diese ultrakurzen Lichtimpulse haben es in den letzten Jahren ermöglicht, diverse elektromagnetische Phänomene auf sehr kurzen Zeitskalen zu untersuchen. Ein solches Phänomen ist das Spin-Rauschen in Orthoferrit, das in dieser Arbeit mittels numerischer Simulationen untersucht wird.

In vielen Messungen ist ein Rauschen in den Messdaten ein unbequemer Begleiter, der immer dabei ist und bei der Auswertung von den \enquote{richtigen Daten} entfernt wird. 
Es gibt gar eine ganze Industrie, die sich mit der Entwicklung von Algorithmen beschäftigt, die Rauschen aus Daten entfernen. Unter anderem bei digitalen Bildaufnahmen, werden diverse komplexe technische Verfahren angewendet, um das Rauschen aus den Bildern zu entfernen. Auch in elektrischen Leitern treten bei endlichen Temperaturen diverse Rauschphänomene auf, die unter anderem Störungen in der Signalübertragung verursachen.\\ 
\todo{ausführungen aus \enquote{the noise is the signal} \cite{noise-is-signal}}

Der DFG-Sonderforschungsbereich 1432 \cite{sfb-1432} beschäftigt sich unter anderem auch mit diversen Rauscheffekten in verschiedenen Systemen.
Hier ist es, aber anders\todo{anders davor noch ausführungen aus \cite{noise-is-signal}}: Das Rauschen ist das interessante Signal, das es zu untersuchen gilt. Im Rahmen der Publikation \enquote{Ultrafast spontaneous spin switching in an antiferromagnet} \cite{weiss-ultrafast} wurde ein Telegrafenrauschen der Magnetisierung (Spinfluktuationen) in Orthoferrit auf einer Picosekundenskala beobachtet. Dieses Telegrafenrauschen ist das schnellste, das bisher beobachtet wurde.\\
Dieses Telegrafenrauschen tritt in der Nähe der Reorientierungstemperatur des schwachen Antiferromagneten auf.\todo{ausführen}\\

Wie bereits der Titel verrät sind diese Spinfluktuationen in \ce{Sm_{0.7}Er_{0.3}FeO3} hier von besonderem Interesse.
In dieser Arbeit sollen die im Rahmen der Masterarbeit von Julius Schlegel \cite{schlegel-master} entstandenen Simulationsdaten von Spin-Rauschen in Samarium-Erbium-Orthoferrit bei verschiedenen Temperaturen genauer untersucht werden. Dafür soll Das Telegrafenrauschen mittels eines Algorithmus aus dem Rauschsignal extrahiert und charakterisiert werden.\\
Außerdem soll der Einfluss eines externen B-Feldes auf die Eingenschaften des Telegrafenrauschens betrachtet werden.

\todo{Aufbau der Arbeit erläutern?}

% bibliography (temporary)
% \newpage\bibliography{literatur} \todo{comment out before compiling main.tex}

\end{document}
