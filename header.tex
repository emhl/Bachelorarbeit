%----------------------------------------------------------------%
%--------------------------INFORMATIONEN-------------------------%
%----------------------------------------------------------------%
%	Infos gibt es zu jedem Paket auf www.ctan.org
%	Werden bei den Paketen bestimmte Optionen gesetzt, so sind die Wichtigsten erklaert
%	solange sie nicht selbsterklärend sind

%----------------------------------------------------------------%
%--------------------------GRUNDEINSTELLUNGEN--------------------%
%----------------------------------------------------------------%
\documentclass[oneside, ngerman, footinclude=on,
    captions=tableheading,]{scrartcl}
%	'oneside'/'twoside': nicht zwischen linker und rechter Seite unterscheiden (alternativ twoside)
%	'twocolumn': wuerde 2 Spalten auf dem Blatt platzieren
%	'bibliography=totocnumbered': Normal nummeriertes Inhaltsverzeichnis (Kapitelnummer)
%	'listof=totocnumbered': Abbildungs- und Tabellenverzeichnis normal nummeriert (Kapitelnummer)
%	'ngerman' verwendet deutsch als Dokumentensprache (z.B. fuer Sirange)
%	'footinclude=off': Zaehlt Fusszeile zum Rand (vergroessert den Textbereich)
%	'captions=tableheading': Tabellenueberschriften explizit verwenden, erhoeht den Abstand zur Tabelle

\addtolength{\textheight}{4.5cm} % verkleinert fußzeile
\usepackage[ngerman, english, main=ngerman]{babel}

\usepackage[T1]{fontenc}
%	Wie wird Text ausgegeben, d.h. im PDF
\usepackage[utf8]{inputenc}
%	Welche Zeichen 'versteht' LaTeX bei der Eingabe?
\usepackage{lmodern}
%	Laedt Schriften, die geglaettet sind
\usepackage{dsfont}
%   ermöglicht Schreiben der Symbole von Mengen wie den Ganzen Zahlen							
\usepackage{blindtext}
%	Beispieltext, zum Testen geeignet
% kein text rechts neben paragraph
\newcommand{\subsubsubsection}[1]{\paragraph{#1}\mbox{}\\}
\newcommand{\subsubsubsubsection}[1]{\subparagraph{#1}\mbox{}\\}
% \setcounter{secnumdepth}{5}

%----------------------------------------------------------------%
%--------------------------ABSTÄNDE------------------------------%
%----------------------------------------------------------------%
\usepackage[onehalfspacing]{setspace}
%	Für Zeilenabstaende: 'singlespacing' (einfach), 'onehalfspacing' (1.5-fach), 'doublespacing' (2fach)
%\setlength{\parindent}{0cm}						%	Laengenangabe für die Einrueckung der ersten Zeile eines neuen Absatzes.
%\setlength{\parskip}{6pt plus 3pt minus 3pt}		%	Laengenangabe für den Abstand zwischen zwei Absaetzen.
%	Wenn diese beiden Befehle nicht kommentiert sind, wird ein Absatz nicht eingezogen sondern es gibt einen Abstand

\usepackage{xparse} %notwendig für physics

%----------------------------------------------------------------%
%--------------------------MATHE---------------------------------%
%----------------------------------------------------------------%
\usepackage[]{mathtools}
%	Erweiterung von AMSMath, laedt automatisch AMSMath - für viele Mathe-Werkzeuge, 'fleqn' als Option ist für Mathe linksbuendig
\usepackage{amsfonts}
%	Für eine Vielzahl an mathematischen Symbolen

%----------------------------------------------------------------%
%--------------------------KOPF- UND FUSSZEILEN------------------%
%----------------------------------------------------------------%
\usepackage[automark,headsepline=.4pt]{scrlayer-scrpage}
\pagestyle{scrheadings}
\setkomafont{pageheadfoot}{\normalfont\bfseries}
%	Normale Schriftart und Fett für den Seitenkopf
\addtokomafont{pagenumber}{\normalfont\bfseries}
%	Normale Schriftart und Fett für die Seitenzahl
\clearpairofpagestyles
\ohead{\thepage}
%	Rechter Seitenkopf mit Seitenzahl
\ihead{\headmark}
%	Linker Seitenkopf mit section
\ofoot[]{\empty}
%	Leere Fußzeile, ungerade Seiten
%	Definert man oben in der documentclass 'twoside', so wird zwischen geraden und ungeraden Seiten unterschieden (NUR DANN!)

%----------------------------------------------------------------%
%--------------------------BILDER--------------------------------%
%----------------------------------------------------------------%
\usepackage{graphicx}
\usepackage{subcaption}
% \usepackage{subfig}
%	Um Bilder einbinden zu koennen 
\usepackage[dvipsnames,svgnames,table]{xcolor}
%	Farben verwenden, Versch. Farbdefinitionen, Farben in Tabellen (-Reihen, -Spalten)
\usepackage{pdfpages}
%	pdfs importieren
\definecolor{Seeblau100}{RGB}{0,169,224}
%	Uni-Farben, z.B. fuer Tabellen
\definecolor{Seeblau65}{RGB}{89,199,254}
\definecolor{Seeblau35}{RGB}{165,224,254}
\definecolor{Seeblau20}{RGB}{203,237,254}
\definecolor{Seegrau60}{RGB}{102,102,102}
\definecolor{Seegrau40}{RGB}{153,153,153}
\definecolor{Seegrau20}{RGB}{204,204,204}
\definecolor{Seegrau10}{RGB}{230,230,230}
\usepackage{svg}

%----------------------------------------------------------------%
%--------------------------POSITIONIERUNG------------------------%
%----------------------------------------------------------------%
\usepackage{float}

%----------------------------------------------------------------%
%--------------------------LISTEN--------------------------------%
%----------------------------------------------------------------%
\usepackage{enumitem}
%	Um Listen / Aufzaehlungen leichter zu modifizieren
%\setlist{noitemsep}							%	Verringert den Abstand in Aufzaehlungen

%----------------------------------------------------------------%
%--------TABELLEN-/BILDUNTERSCHRIFTEN und NUMMERIERUNG-----------%
%----------------------------------------------------------------%
\addtokomafont{captionlabel}{\bfseries}
%	Abbildung X.Y wir fett geschrieben
\setcapindent{2em}
%	2. Zeile teilweise haengend und eingezogen. Wenn ganz haengend gewuenscht, auskommentieren

\numberwithin{equation}{section}
%	Nummerierung der Gleichungen, Tabellen und Bilder nach der Kapitelnummer
\numberwithin{figure}{section}
\numberwithin{table}{section}

\usepackage{noindentafter}
\NoIndentAfterEnv{figure}
\NoIndentAfterEnv{table}
\NoIndentAfterEnv{tabular}
\NoIndentAfterEnv{align}
\NoIndentAfterEnv{itemize}
%----------------------------------------------------------------%
%--------------------------LITERATURVERZEICHNIS------------------%
%----------------------------------------------------------------%
\usepackage[german]{babelbib}%	Bereitstellung des deutschen Layouts fuer die Bibliography
\bibliographystyle{babalpha}

%----------------------------------------------------------------%
%--------------------------SIUNITX-------------------------------%
%----------------------------------------------------------------%
\usepackage[]{siunitx}
\NewCommandCopy\qnt\qty
\sisetup{locale = DE}
%	Automatische Einstellung der Ausgabe für bestimmte Regionen (UK, US, DE, FR, ZA)

%----------------------------------------------------------------%
%--------------------------URLs / REFs---------------------------%
%----------------------------------------------------------------%
\usepackage[hidelinks]{hyperref}
%	Erweiterte Referenzierung ('hidelinks' verhindert Linien um Links)

\usepackage{cleveref}

%----------------------------------------------------------------%
%--------------------------EIGENE BEFEHLE------------------------%
%----------------------------------------------------------------%
\newcommand{\dif}{\mathrm{d}}
\newcommand{\todo}[1]{{\fontfamily{phv}\selectfont\textcolor{red}{\textbf{ToDo:}~#1}}}

%% Pakete für Umlaute, Rechtschreibung und Orthographie
%%
%\usepackage[english,ngerman]{babel} % Neue deutsche Rechtschreibung, für primär englischen Text Reihenfolge tauschen
%% Seit TeXLive 2018 ist utf8 das default inputenc
% \usepackage[utf8]{inputenc} % Eingabeencoding (nicht verwechseln mit Dateiencoding im Editor),
%%                             ermöglicht direkte Verwendung von Umlauten im TeX
%%
\usepackage[T1]{fontenc} % Neues LaTeX Schriftart-Encoding für korrekte Darstellung von Umlauten im PDF 
%%                       % z.B. richtige Glyphen anstatt Kombination aus ¨ und Buchstaben
%%                       % Es wird nun UTF8 für Glyphen verwendet
\usepackage{textcomp} % Für richtige Upperquotes in Programmcode-Listings ohne Probleme bei Copy&Paste
\usepackage[babel]{csquotes} % Richtige Anführungszeichen leicht gemacht: \enquote{Text in Anführungszeichen} statt Kombinationen aus `` und ''.
\usepackage{epigraph} % Blockzitate (wie sie in vielen Büchern am Anfang gefunden werden)

\usepackage{physics} % shortcuts für physikalische gleichungen  https://mirror.clientvps.com/CTAN/macros/latex/contrib/physics/physics.pdf
\usepackage[version=4]{mhchem} % chemische Formeln
\usepackage{csvsimple} % Tabellen aus csv importieren.

%%%%%%%%%%%%%%%%%%%%%%%%%
% Datum, Autor, etc.    %
%%%%%%%%%%%%%%%%%%%%%%%%%

\def\docDate{}
\renewcommand{\date}[1]{
    \def\docDate{#1}
}

\def\docYear{}
\renewcommand{\year}[1]{
    \def\docYear{#1}
}

\def\docAuthor{}
\renewcommand{\author}[1]{
    \def\docAuthor{#1}
}

\def\docTitle{}
\renewcommand{\title}[1]{
    \def\docTitle{#1}
}

\def\docSubtitle{}
\renewcommand{\subtitle}[1]{
    \def\docSubtitle{#1}
}

\def\docThesisType{}
\newcommand{\thesisType}[1]{
    \def\docThesisType{#1}
}

\def\docUnisection{}
\newcommand{\unisection}[1]{
    \def\docUnisection{#1}
}

\def\docDepartment{}
\newcommand{\department}[1]{
    \def\docDepartment{#1}
}

\def\docSupervisorOne{}
\newcommand{\supervisorOne}[1]{
    \def\docSupervisorOne{#1}
}

\def\docSupervisorTwo{}
\newcommand{\supervisorTwo}[1]{
    \def\docSupervisorTwo{#1}
}